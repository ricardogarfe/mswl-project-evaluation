\documentclass[11pt]{scrartcl}

\title{\textbf{Project Evaluation}}
\subtitle{QUALOSS}
\author{Daniel Izquierdo}
\date{\today}

\begin{document}

\maketitle

\section{Introduction}

QuaLOSS; Quality of Libre Open Source Software

Not too many quality models are focused on FLOSS so far and if so, they are 
still too light weight : OpenBRR / QSoS

Una empresa necesitaría gente especializada para poder aplicar un modelo de 
calidad, pero únicamente lo pueden hacer las grandes empresas por ello una 
pequeña enpresa pueded apoyarse en los modelos de calidad existentes y 
aplicarlos adecuando las características en la empresa.

Universia - red interuniversitaria - Si era o no FLOSS.
Igalia - Desarrollo de software - Licencia y comunidad.

A partir de la información obtenida de las distintas empresas se unificaron los 
componentes de las métricas a tratas para los proyectos FLOSS.

TODO: imdea software
      Programa Marco

\section{Métricas}

\begin{itemize}
    \item Test. 
    \item Nuevos miembros de la comunidad. Evolution of new core contributing. 
    \item iwa4 proportion of files maintained by a single committer,
    \item sra4 number of new core committers in the year
    \item sra5 number of core committers that quit in the year
    \item sra6 difference between sra4 and sra5.
    \item Aportes de código fuente.
    \item Vida del primer desarrollador con respecto al proyecto si está y si el proyecto se mantiene sin su marcha.
\end{itemize}

\section{}

\end{document}
