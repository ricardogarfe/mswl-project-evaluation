\documentclass[11pt]{scrartcl}

\title{\textbf{Project Evaluation}}
\subtitle{Introduction to Quality in Libre Software}
\author{Pedro coca}
\date{\today}

\begin{document}

\maketitle

\section{Introduction}

Vamos a dar un repaso a las cualidades que se han de mirar a la hora de evaluar proyectos de software libre como; soporte, madurez, número de releases, origen del proyecto...

Estudiaremos las métricas Qualos y OpenBRR.

\section{StackSherpa}

Empresa relacionada con Saas, tecnología cloud y Paas (plataforma como servicio Cloudfoundry).

\subsection{Cloud computing}

Cloud computing, servicio bajo demanda de capacidad de c'omputo.

\begin{itemize}
    \item Public clouds, amazon ws.
    \item Private clouds, locales en el servidor de la empresa.
    \item Community clouds, nube privada, como un bus.
    \item Hybrid clouds, privada hasta que excede la capacidad de c\'alculo comienza a consumir recursos p\'ublicos.
\end{itemize}

\subsection{SPI}

SPI:
\begin{itemize}
    \item Saas - Gmail.
    \item Paas - CLoudfoundry.
    \item Iaas - SO, Virtualization, Dedicated resources, Hardware - (Amazon WS)
\end{itemize}

Ley de Moore, computaci'on y almacenamiento.

\section{Iaas}

OpenNebula\footnote{http://opennebula.org/}; proyecto de la Universidad Complutense de Madrid. Caso de \'exito: CERN.

\section{Evaluation}

Que ha pasado con otros proyectos en manos de las empresas que est\'an detr\'as.

Como evaluar los proyectos en cada caso por necesidad.

\section{OpenBRR}

OpenBRR para la adopci\'on de un proyecto FLOSS que ofrece un modelo \emph{Complete Simple Adaptable Consisten} - CSAC. \\
Antes de OpenBRR y OpenModel, se pod\'ian pasar por alto algunos casos que dejar\'ian el mismo como no reutilizable.

Consta de cuatro fases:
\begin{itemize}
    \item \emph{Quick Assesmente Filter}: por ejemplo aplicar el filtro de las licencias.
    \item TBC
    \item TBC
    \item TBC
\end{itemize}

OpenBRR sugiere seleccionar 7 o menos m\'etricas m\'as relevantes para nuestro caso para organizar por categor\'ias y definir un peso a cada una de ellas. \\

\section{Qualos}

\section{IaaS}

\end{document}
