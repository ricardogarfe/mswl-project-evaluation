\documentclass[xcolor=dvipsnames]{beamer}
\usecolortheme[named=Blue]{structure}
\usetheme{Warsaw}
\usepackage{beamerthemesplit}
\usepackage{latexsym}
\usepackage{eurosym}
\usepackage{ae,aecompl}
\usepackage{graphicx}
\usepackage{amsfonts}
\usepackage{tikz}

\usetheme{Darmstadt}

\title{Metrics for SCM Analysis}
\centering\titlegraphic{\includegraphics[width=3.3in]{images/subversion_logo.png}}
\author{Ricardo Garc\'ia Fern\'andez,\\
        Project Evaluation,\\
        Universidad Rey Juan Carlos I.}
\date{\today}

\begin{document}


\frame{\titlepage
\begin{flushright}
{\tiny
(cc) 2012 Ricardo Garc\'ia Fern\'andez\\
    Este obra est\'a bajo una licencia de Creative Commons Reconocimiento 3.0 Unported.
    To view a copy of full license, see http://creativecommons.org/licenses/by/3.0/
}
\end{flushright}
}

\section[Index]{}
\begin{frame}[allowframebreaks]
\tableofcontents
\end{frame}

\section{VCS}
\begin{frame}[allowframebreaks]
\frametitle{Version Control System}

\textbf{VCS}: Version Control System. Un Sistema de Control de Versiones es una herramienta para la gesti\'on de los archivos y su ciclo de vida dentro de un proyecto. Gestiona todas las acciones que se realizan sobre ellos, crear, guardar, copiar, borrar, mover a través de una base de datos, creando un hist\'orico y ofreciendo una gesti\'on \'agil de los recursos.

\end{frame}

\section{Subversion}
\begin{frame}[allowframebreaks]
\frametitle{Subversion}
\textbf{Subversion}: es un Sistema de Control de Versiones de c\'odigo abierto.
Fue creado en el a\~no 2000 por la empresa Collabnet\footnote{http://www.collab.net/} perteneciente a la fundaci\'on Apache.
Es el VCS m\'as utilizado en los proyectos relacionados con el desarrollo de proyectos de software libre y proyectos privados.
\end{frame}

\section{Metrics}
\begin{frame}[allowframebreaks]
\frametitle{Metrics}
\textbf{M\'etricas}: Son un conjunto de cualidades evaluables que aportan informaci\'on cuantificable como resultado del an\'alisis.

\par Se ha de definir un conjunto de m\'etricas para la evaluaci\'on del proyecto Subversion y poder cuantificar con una nota el resultado final
\end{frame}

\begin{frame}[allowframebreaks]
\frametitle{OpenBRR Model}

\textbf{OpenBRR} nace como un proyecto de unificaci\'on de m\'etricas para los proyectos de Software. Este modelo se considera libre ya que est\'a orientado a los proyectos de Software Libre (FLOSS). Naci\'o en el a\~no 2005 con ese fin pero, como en su p\'agina web se indica, no ha llegado a crear una gran comunidad alrededor.

\end{frame}

\begin{frame}[allowframebreaks]
\frametitle{Metrics chosen}

Se han elegido un \textbf{conjunto de m\'etricas} para representar los valores m\'as importantes relacionados con la gesti\'on de los archivos. Dividido en varias categor\'ias:

\begin{tabular}{|l|}
    \hline {\bf Category}\\
    \hline Initial Analysis\\
    \hline Functionality\\
    \hline Usability\\
    \hline Robustness\\
    \hline Development\\
    \hline Community\\
    \hline Documentation\\
    \hline
\end{tabular}

\end{frame}

\begin{frame}[allowframebreaks]
\frametitle{Weight \& Punctuation}

\par Despu\'es del an\'alisis de los requerimientos y puntuaci\'on de las m\'etricas se procede a dar un peso espec\'ifico a cada secci\'on definida en el documento.

\begin{tabular}{|l|l|}
    \hline {\bf Category} & {\bf Weight}\\
    \hline Initial Analysis	 & 13,00\%\\
    \hline Functionality & 12,00\%\\
    \hline Usability & 14,00\%\\
    \hline Robustness & 17,00\%\\
    \hline Development & 15,00\%\\
    \hline Community & 19,00\%\\
    \hline Documentation & 10,00\%\\
    \hline
\end{tabular}

Veamos paso a paso el an\'alisis de las m\'etricas por orden de importancia con respecto al \% dedicado.

\end{frame}

\begin{frame}[allowframebreaks]
\frametitle{Community}

\par \textbf{Community} se divide en las siguientes sub-m\'etricas.

\begin{tabular}{|l|l|l|}
    \hline {\bf Category} & {\bf Weight} & {\bf Punctuation}\\
    \hline Avg commits per month, last six months & 20,00\% & 2\\
    \hline Avg monthly volume of general mailing lists \\during the last six months & 15,00\% & 2\\
    \hline Developers that left the project and those\\ that started to participate (last year)\\ (and also for the  core team) & 15,00\% & 5\\
    \hline Knowledge concentration (territoriality) & 10,00\% & 0\\
    \hline Original developer/team & 40,00\% & 5\\
    \hline
\end{tabular}

Sumatorio de las puntuaciones de 1 a 5 ponderadas con respecto al \% que se otorga a cada secci\'on: 

\end{frame}

\begin{frame}[allowframebreaks]
\frametitle{Robustness}

\par \textbf{Robustness} se divide en las siguientes sub-m\'etricas.

\begin{tabular}{|l|l|l|}
    \hline {\bf Category} & {\bf Weight} & {\bf Punctuation}\\
    \hline Mature and stable version & 70,00\% & 5\\
    \hline Num. point/patch releases last year & 3,2\%\\
    \hline
\end{tabular}

Puntuaciones de 1 a 5 ponderadas con respecto al \% que se otorga a cada secci\'on.

\end{frame}

\begin{frame}[allowframebreaks]
\frametitle{Development}

\par \textbf{Development} se divide en las siguientes sub-m\'etricas.

\begin{tabular}{|l|l|l|}
    \hline {\bf Category} & {\bf Weight} & {\bf Punctuation}\\
    \hline API	 & 55,00\% & 4,4\\
    \hline Several levels of documentation & 45,00\% & 4,3\\
    \hline
\end{tabular}

Puntuaciones de 1 a 5 ponderadas con respecto al \% que se otorga a cada secci\'on.

\end{frame}

\begin{frame}[allowframebreaks]
\frametitle{Usability}

\par \textbf{Usability} se divide en las siguientes sub-m\'etricas.

\begin{tabular}{|l|l|l|}
    \hline {\bf Category} & {\bf Weight} & {\bf Punctuation}\\
    \hline UI client	 & 30,00\% & 4,6\\
    \hline Easy learning & 70,00\% & 4,7\\
    \hline
\end{tabular}

Puntuaciones de 1 a 5 ponderadas con respecto al \% que se otorga a cada secci\'on.

\end{frame}

\begin{frame}[allowframebreaks]
\frametitle{Initial Analysis}

\par \textbf{Initial Analysis} se divide en las siguientes sub-m\'etricas.

\begin{tabular}{|l|l|l|}
    \hline {\bf Category} & {\bf Weight} & {\bf Punctuation}\\
    \hline License	 & 30,00\% & 4,2\\
    \hline Time for setup & 70,00\% & 4\\
    \hline
\end{tabular}

Puntuaciones de 1 a 5 ponderadas con respecto al \% que se otorga a cada secci\'on.

\end{frame}

\begin{frame}[allowframebreaks]
\frametitle{Functionality}

\par \textbf{Functionality} se divide en las siguientes sub-m\'etricas.

\begin{tabular}{|l|l|l|}
    \hline {\bf Category} & {\bf Weight} & {\bf Punctuation}\\
    \hline End user UI experience & 40,00\% & 5\\
    \hline Backup and restore & 30,00\% & 5\\
    \hline Profile management & 30,00\% & 2,9\\
    \hline
\end{tabular}

Puntuaciones de 1 a 5 ponderadas con respecto al \% que se otorga a cada secci\'on.

\end{frame}

\begin{frame}[allowframebreaks]
\frametitle{Documentation}

\par \textbf{Documentation} se divide en las siguientes sub-m\'etricas.

\begin{tabular}{|l|l|l|}
    \hline {\bf Category} & {\bf Weight} & {\bf Punctuation}\\
    \hline Several levels of documentation & 100,00\% & 4,15\\
    \hline
\end{tabular}

Puntuaciones de 1 a 5 ponderadas con respecto al \% que se otorga a cada secci\'on.

\end{frame}

\section{Results}
\begin{frame}[allowframebreaks]
\frametitle{Analysis Results}

Podemos apreciar que el resultado obtenido es una puntuaci\'on de \textbf{4,2 sobre 5} que se traduce a \textbf{83,97 sobre 100}.

\begin{tabular}{|l|l|l|l|}
    \hline {\bf Category} & {\bf Weight} & {\bf Unweighted Rating} & {\bf Weighted Rating}\\
    \hline Initial Analysis	 & 13,00\% & 4,14 & 0,54 \\
    \hline Functionality & 12,00\% & 4,37 & 0,52\\
    \hline Usability & 14,00\% & 4,67 & 0,65\\
    \hline Robustness & 17,00\% & 4,46 & 0,76\\
    \hline Development & 15,00\% & 4,36 & 0,65\\
    \hline Community & 19,00\% & 3,45 & 0,66\\
    \hline Documentation & 10,00\% & 4,15 & 0,42\\
    \hline
\end{tabular}

\end{frame}

\section{Tools}
\begin{frame}[allowframebreaks]
\frametitle{Pick up Information}

\par Para este an\'alisis se han utilizado varias herramientas para conseguir la informaci\'on necesaria.

\begin{itemize}
    \item \emph{Web de Subversion} - Listas de correo, usuarios, repositorio, documentaci\'on, foros.
    \item \emph{CVSAnaly} - Herramienta para obtener informaci\'on de un repositorio de c\'odigo. Crea una base de datos con toda la informaci\'on referente al repositorio, usuarios, commits, ficheros.
    \item \emph{libcvsanaly2} - Librer\'ia para generar informes a partir de la base de datos creada por CVSAnaly2 con respecto a las acciones sobre los ficheros de un repositorio.
    \item \emph{ohloh} - Es un repositorio de repositorios (RoR) donde muestra informaci\'on sobre los proyectos que alberga.

\end{itemize}

\end{frame}

\section{Bibliography}
\begin{frame}[allowframebreaks]
\frametitle{Bibliography}
\begin{thebibliography}{9}

\bibitem{openbrr}
    OpenBRR,\\
    http://www.openbrr.org

\bibitem{subversion-site}
  Subversion Tigris,\\
  http://subversion.tigris.org/

\bibitem{subversion-apache}
  Subversion Apache,\\
  http://subversion.apache.org/

\bibitem{cvsanaly2}
    CVSAnaly\\
    https://github.com/MetricsGrimoire/CVSAnalY

\bibitem{libcvsanaly2}
    libcvsanaly2\\
    http://git.libresoft.es/libcvsanaly2

\bibitem{ohloh}
    Ohloh.net\\
    http://www.ohloh.net

\end{thebibliography}
\end{frame}

\end{document}
