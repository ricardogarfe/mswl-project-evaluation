\documentclass[xcolor=dvipsnames]{beamer}
\usecolortheme[named=Blue]{structure}
\usetheme{Warsaw}
\usepackage{beamerthemesplit}
\usepackage{latexsym}
\usepackage{eurosym}
\usepackage{ae,aecompl}
\usepackage{graphicx}
\usepackage{amsfonts}
\usepackage{tikz}

\usetheme{Darmstadt}

\title{Metrics for SCM Analysis}
\centering\titlegraphic{\includegraphics[width=3.3in]{images/subversion_logo.png}}
\author{Ricardo Garc\'ia Fern\'andez,\\
        Project Evaluation,\\
        Universidad Rey Juan Carlos I.}
\date{\today}

\begin{document}


\frame{\titlepage
\begin{flushright}
{\tiny
(cc) 2012 Ricardo Garc\'ia Fern\'andez\\
    Este obra est\'a bajo una licencia de Creative Commons Reconocimiento 3.0 Unported.
    To view a copy of full license, see http://creativecommons.org/licenses/by/3.0/
}
\end{flushright}
}

%% \usebackgroundtemplate{
%%   \parbox[c][\paperheight][c]{\paperwidth}{\centering\includegraphics[width=5.5in]{logo_springsource_community.png}}
%% }

\section[Index]{}
\begin{frame}[allowframebreaks]
\tableofcontents
\end{frame}

\section{VCS}
\begin{frame}[allowframebreaks]
\frametitle{Version Control System}

\textbf{VCS}: Version Control System. Un Sistema de Control de Versiones es una herramienta para la gesti\'on de los archivos y su ciclo de vida dentro de un proyectol. Gestiona todas las acciones que se realizan sobre ellos, crear, guardar, copiar, borrar, mover en una base de datos, creando un hist\'orico y ofreciendo una gesti\'on \'agil de los recursos.

Vamos a implantar un Control de Versiones para el trabajo en nuestra empresa de desarrollo de software mediante el an\'alisis de las m\'etricas escogidas.

\end{frame}

\section{Subversion}
\begin{frame}[allowframebreaks]
\frametitle{Subversion}
\textbf{Subversion}: es un Sistema de Control de Versiones de c\'odigo abierto.
Fue creado en el a\~no 2000 por la empresa Collabnet\footnote{http://www.collab.net/}.
Es el VCS m\'as utilizado en los proyectos de relacionados con el desarrollo de software.
\end{frame}

\section{Metrics}
\begin{frame}[allowframebreaks]
\frametitle{Metrics}
\begin{itemize}
    \item \emph{Get Started}\footnote{http://www.springsource.org/get-started}
    \item \emph{Get Involved}\footnote{http://www.springsource.org/get-involved}
\end{itemize}
\end{frame}

\begin{frame}[allowframebreaks]
\frametitle{OpenBRR Model}
\begin{itemize}
    \item \emph{Start a Tutorial} - Tutoriales de uso.
    \item \emph{Grab a Code Sample} - Ejemplos funcionales.
    \item \emph{Ask a Question (Forums)} - Foro, activo e importante.
    \item \emph{Take a Class (Training)} - "Universidad" de Spring para aprender a codificar.
    \item \emph{Read the Documentation} - Important\'isimo apartado, no s\'olo leer si no, saber utilizar la documentaci\'on.
    \item \emph{Video Instruction} - V\'ideos en los que muestra las herramientas y su uso en funcionamiento.
\end{itemize}
\end{frame}

\begin{frame}[allowframebreaks]
\frametitle{Choosed Metrics}
\begin{itemize}
    \item \emph{Join the conversarion} - Informaci\'on del d\'ia a d\'ia.
    \item \emph{Help other users (And get help when you need it too)} - Uso de los foros y otros canales.
    \item \emph{Report issues} - Informa de errores o mejoras.
    \item \emph{Track the latest features and test them out} - Uso activo del JIRA para poder probar las nuevas caracter\'isticas o posibles errores.
    \item \emph{Contribute code} - Formar parte del grupo de usuarios que aportan c\'odigo a Spring Framework.
    \item \emph{Attend (or give) a talk at a local user group} - Anima al desarrollador a ir a charlas sobre Spring o a ser el ponente de la misma charla en tus grupos cercanos de desarrolladores.
\end{itemize}
\end{frame}

\begin{frame}[allowframebreaks]
\frametitle{Weight \& Punctuation}
\begin{itemize}
    \item \emph{Join the conversarion} - Informaci\'on del d\'ia a d\'ia.
    \item \emph{Help other users (And get help when you need it too)} - Uso de los foros y otros canales.
    \item \emph{Report issues} - Informa de errores o mejoras.
    \item \emph{Track the latest features and test them out} - Uso activo del JIRA para poder probar las nuevas caracter\'isticas o posibles errores.
    \item \emph{Contribute code} - Formar parte del grupo de usuarios que aportan c\'odigo a Spring Framework.
    \item \emph{Attend (or give) a talk at a local user group} - Anima al desarrollador a ir a charlas sobre Spring o a ser el ponente de la misma charla en tus grupos cercanos de desarrolladores.
\end{itemize}
\end{frame}

\section{Tools}
\begin{frame}[allowframebreaks]
\frametitle{Pick up Information}
Dise\~no de la comunicaci\'on dentro de la comunidad en el que intervienen todos los roles.
\begin{itemize}
    \item Usuario - Documentaci\'on - Foro - JIRA - Votaci\'on - Resoluci\'on.
\end{itemize}
\emph{Todo el proceso viene apoyado a trav\'es del seguimiento de la comunidad mediante su participaci\'on.}
\end{frame}

\begin{frame}[allowframebreaks]
\frametitle{Tools}
Dise\~no de la comunicaci\'on dentro de la comunidad en el que intervienen todos los roles.
\begin{itemize}
    \item Usuario - Documentaci\'on - Foro - JIRA - Votaci\'on - Resoluci\'on.
\end{itemize}
\emph{Todo el proceso viene apoyado a trav\'es del seguimiento de la comunidad mediante su participaci\'on.}
\end{frame}

\section{Bibliography}
\begin{frame}[allowframebreaks]
\frametitle{Bibliography}
\begin{thebibliography}{9}

\bibitem{subversion-site}
  Subversion Tigris,\\
  http://subversion.tigris.org/

\bibitem{subversion-apache}
  Subversion Apache,\\
  http://subversion.apache.org/

\end{thebibliography}
\end{frame}

\end{document}
